\documentclass[a4paper,10pt]{beamer}
\usepackage[T1]{fontenc}
\usepackage[utf8]{inputenc}
\usepackage{lmodern}
\usepackage[francais]{babel}
\usepackage{amsmath}
\usepackage{amsfonts}
\usepackage{amssymb}
\usepackage{xcolor}
\usepackage{graphicx}
\numberwithin{equation}{section}
\usetheme{Warsaw}
\usefonttheme{serif}
\newtheorem{Pro1}{Proposition}[section]

\title[\sc Assimilation de données ]{\sc Assimilation de données et Réduction d'ordre:\\ PBDW et GEIM}
\author[\sc I.NIAKH]{\sc NIAKH IDRISSA}
\institute[\sc UNISTRA]{\sc Sous la direction de Christophe Prud'homme }
\date[08-06-18]{08 Juin 2018}
\AtBeginSection[]
{
\begin{frame}{Plan}
\tableofcontents[currentsection]
\end{frame}
}

\begin{document}
	
\begin{frame}
\begin{figure}
    \includegraphics[width=4cm]{/home/niakh/Bureau/Master_CSMI/CSMI/Master_CSMI/S2/Stage/unistra.jpg}
\end{figure}
\titlepage
\end{frame}

\begin{frame}
{\contentsname}
\tableofcontents
\end{frame}
\section{\sc Introduction}
\AtBeginSection[]
{
\begin{frame}{Plan}
\tableofcontents[currentsection]
\end{frame}
}


\begin{frame}
\frametitle{\sc Assimilation de données (AD)}
\begin{block}{\sc Définition}
	Couplage modèle mathématique et observations expérimentales.
	\pause
\end{block}
\begin{block}{\sc Méthodes d'assimilation de données}
	\pause
	\begin{itemize}[<+->]
		\item Méthodes séquentielles
		\item Méthodes inverses
		\item Méthodes variationnelles
	\end{itemize}
\end{block}
\end{frame}


\begin{frame}
\frametitle{\sc Réduction d'ordre(RO)}
\begin{block}{\sc Définition}
	Méthode de réduction du coût de calcul d'une simulation.
	\pause
\end{block}
\begin{block}{\sc Méthode de réduction d'ordre}
	\pause
	\begin{itemize}[<+->]
		\item Méthodes de substitution
		\item Méthodes bases réduites 
		\item Méthodes de décomposition
		\item Méthodes d'interpolation empirique
	\end{itemize}
\end{block}

\end{frame}

\begin{frame}{\sc Couplage AD-RO}
\begin{block}{\sc Méthodes}
	\pause
	\begin{itemize}[<+->]
		\item Parametrized Background Data Weak(PBDW)
		\item Generalized Empirical Interpolation Method(GEIM)
	\end{itemize}
\end{block}
\end{frame}
\section{\sc Contexte et Problématique}

\AtBeginSection[]
{
\begin{frame}{Plan}
\tableofcontents[currentsection]
\end{frame}
}

\begin{frame}
\begin{block}{\sc Contexte}
	\pause
	\begin{itemize}[<+->]
		\item Modèle imprécis
		\item Simulation coûteuse
	\end{itemize}
\pause
\end{block}

\begin{block}{\sc Problématique}
	\pause
	\begin{itemize}[<+->]
		\item Comment peut-on corriger le modèle en utilisant les observations expérimentales?
		\item Comment peut-on réduire le temps de simulation tout en restant proche de la vraie solution?
	\end{itemize}
\end{block}
\end{frame}

\section{\sc PBDW}

\AtBeginSection[]
{
\begin{frame}{Plan}
\tableofcontents[currentsection]
\end{frame}
}

\section{\sc GEIM}
\AtBeginSection[]
{
\begin{frame}{Plan}
\tableofcontents[currentsection]
\end{frame}
}
\begin{frame}

\end{frame}

\section{\sc Complémentarité PBDW et GEIM}
\AtBeginSection[]
{
\begin{frame}{Plan}
\tableofcontents[currentsection]
\end{frame}
}
\section{\sc Autres types de méthodes}
\AtBeginSection[]
{
\begin{frame}{Plan}
\tableofcontents[currentsection]
\end{frame}
}


\end{document}